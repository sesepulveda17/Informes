%!TeX jobName=challenges/stego/beatles
%\nofiles
% Created by Bonita Graham
% Last update: February 2019 By Kestutis Bendinskas
%https://es.overleaf.com/gallery/tagged/academic-journal
% Authors:
% Please do not make changes to the preamble until after the solid line of %s.

\documentclass[letterpaper,10pt]{article}
\usepackage[spanish,es-noshorthands]{babel}
\usepackage[latin1,utf8]{inputenc} % Codificación UTF-8

\usepackage[explicit]{titlesec}
\setlength{\parindent}{0pt}
\setlength{\parskip}{1em}
\usepackage{hyphenat}
\usepackage{ragged2e}
%Para colocar puntos en los itemes especiales
\usepackage{pifont}
\RaggedRight

% These commands change the font. If you do not have Garamond on your computer, you will need to install it.
%\usepackage{garamondx}
\usepackage[T1]{fontenc}
\usepackage{amsmath, amsthm}
\usepackage{graphicx}
\usepackage{caption}

% This adjusts the underline to be in keeping with word processors.
\usepackage{soul}
\setul{.6pt}{.4pt}
%This is for the color for the codign
\usepackage[dvipsnames]{xcolor}
%http://latexcolor.com/
\definecolor{almond}{rgb}{0.94, 0.87, 0.8}
\definecolor{airforceblue}{rgb}{0.36, 0.54, 0.66}
\usepackage{listings}

\lstset{
  backgroundcolor=\color{almond}, %color de fondo
  breaklines=true,
  captionpos=b,     % Establece la posición de la leyenda del cuadro de código
  basicstyle=\sffamily,
  %basicstyle=\footnotesize,
  showstringspaces=false,
  commentstyle=\color{red},
  keywordstyle=\color{blue}
}

%https://www.overleaf.com/learn/latex/Hyperlinks
%Styles and colours
\usepackage{hyperref}
\hypersetup{
    colorlinks=true,
    linkcolor=blue,
    filecolor=magenta,
    urlcolor=ForestGreen,
}
\urlstyle{same}


% The following sets margins to 1 in. on top and bottom and .75 in on left and right, and remove page numbers.
\usepackage{geometry}
\geometry{vmargin={1in,1in}, hmargin={.75in, .75in}}
\usepackage{fancyhdr}
\pagestyle{fancy}
\pagenumbering{gobble}
\renewcommand{\headrulewidth}{0.0pt}
\renewcommand{\footrulewidth}{0.0pt}

% These Commands create the label style for tables, figures and equations.
\usepackage[labelfont={footnotesize,bf} , textfont=footnotesize]{caption}
\captionsetup{labelformat=simple, labelsep=period}
\newcommand\num{\addtocounter{equation}{1}\tag{\theequation}}
\renewcommand{\theequation}{\arabic{equation}}
\makeatletter
\renewcommand\tagform@[1]{\maketag@@@ {\ignorespaces {\footnotesize{\textbf{Equation}}} #1.\unskip \@@italiccorr }}
\makeatother
\setlength{\intextsep}{10pt}
\setlength{\abovecaptionskip}{2pt}
\setlength{\belowcaptionskip}{-10pt}

\renewcommand{\textfraction}{0.10}
\renewcommand{\topfraction}{0.85}
\renewcommand{\bottomfraction}{0.85}
\renewcommand{\floatpagefraction}{0.90}

% These commands set the paragraph and line spacing
\titleformat{\section}
  {\normalfont}{\thesection}{1em}{\MakeUppercase{\textbf{#1}}}
\titlespacing\section{0pt}{0pt}{-10pt}
\titleformat{\subsection}
  {\normalfont}{\thesubsection}{1em}{\textit{#1}}
\titlespacing\subsection{0pt}{0pt}{-8pt}
\renewcommand{\baselinestretch}{1.15}

% This designs the title display style for the maketitle command
\makeatletter
\newcommand\sixteen{\@setfontsize\sixteen{16pt}{6}}
\renewcommand{\maketitle}{\bgroup\setlength{\parindent}{0pt}
\begin{flushleft}
\vspace{-.375in}
\sixteen\bfseries \@title
\medskip
\end{flushleft}
\textsc{\@author}\\
\textit{\today}
\egroup}
\makeatother

% This styles the bibliography and citations.
%\usepackage[biblabel]{cite}
\usepackage[sort&compress]{natbib}
\setlength\bibindent{2em}
\makeatletter
\renewcommand\@biblabel[1]{\textbf{#1.}\hfill}
\makeatother
\renewcommand{\citenumfont}[1]{\textbf{#1}}
\bibpunct{}{}{,~}{s}{,}{,}
\setlength{\bibsep}{0pt plus 0.3ex}

\renewcommand*{\lstlistingname}{Código}  %CAMBIA EL TITULO DE LISTING EN EL CODIGO


%%%%%%%%%%%%%%%%%%%%%%%%%%%%%%%%%%%%%%%%%%%%%%%%%

% Authors: Add additional packages and new commands here.
% Limit your use of new commands and special formatting.

% Place your title below. Use Title Capitalization.
\title{Write-up: HackTheBox - Stego - Beatles}

% Add author information below. Communicating author is indicated by an asterisk, the affiliation is shown by superscripted lower case letter if several affiliations need to be noted.
\author{Sebastián Sepúlveda @piblack}


\pagestyle{empty}
\begin{document}

% Makes the title and author information appear.
\vspace*{.01 in}
\maketitle
\vspace{.12 in}

% Start the main part of the manuscript here.
% Comment out section headings if inappropriate to your discipline.
% If you add additional section or subsection headings, use an asterisk * to avoid numbering.

\textbf{Información:} Categoria Stego, puntuacion 20.

\textbf{Descripción:} John Lennon send a secret message to Paul McCartney about the next music tour of Beatles... Could you find the message and sumbit the flag?

\section*{WriteUp}

El desafio nos entrega 2 archivos, un \texttt{.txt} y un \texttt{.zip}. El archivo de texto se ve algo así
\begin{lstlisting}[language=bash,caption={Revision de mensaje}]
root@retro:~/Descargas/beatles# cat m3ss@g#_f0r_pAuL
Url Cnhy,
Zl Sbyqre unf cnffcuenfr jvgu sbhe (4) punenpgref.
Pbhyq lbh spenpx vg sbe zr???
V fraq lbh n zrffntr sbe bhe Gbhe arkg zbagu...
Qba'g Funer vg jvgu bgure zrzoref bs bhe onaq...
-Wbua Yraaba
CF: Crnpr naq Ybir zl sevraq... Orngyrf Onaq sbe rire!
\end{lstlisting}

Hacemos el tipico paso para descifrar el cifrado que se ocupa. Ocupamos \href{http://mtk911.cf/cipher/}{MTH911} para saber de que se trata.

Obtenemos que es un cifrado Patristocrat, así que nos movemos al \href{https://bionsgadgets.appspot.com/ww_forms/aristo_pat_web_worker3.html}{Cipher Patristocrat} para descifrarlo. Nos devuelve:

\textsc{
Hey Paul,\\
My Folder has passphrase with four (4) characters.\\
Could you fcrack it for me???\\
I send you a message for our Tour next month...\\
Don't Share it with other members of our band...\\
-John Lennon\\
PS: Peace and Love my friend... Beatles Band for ever!}
\par

Una gran ayuda para descifrar la clave del archivo \texttt{.zip}. Nos indican que tenemos que obtener el password utilizando fcrackzip por tanto seguimos con la operación.
Nos lanzará un mensaje así:

\texttt{
fcrackzip -u -D -p /usr/share/wordlists/rockyou.txt BAND.zip
PASSWORD FOUND!!!!: pw == pass
}
\newpage
Lo ocupamos para descifrar el \texttt{.zip} y obtenemos la imagen del disco \texttt{HELP!} de los Beatles:

\begin{figure}[h]
  \centering
  \includegraphics[scale=0.2]{images/beatles/help}
  \captionof{figure}{Imagen obtenido después de descomprimir el zip}
  \label{fig:archivo}
\end{figure}

Luego tenemos que saber que es lo que tiene escondido esta imagen. Ocupamos Strings, Binwalk, pero no son los que nos sirven. Finalmente ocupamos \texttt{steghide extract -sf BAND.JPG}, lo que nos pide un salvaconducto, usamos THEBEATLES:

\textsc{
>> Anotar salvoconducto:\\
>> anot- los datos extra-dos e/"testabeatle.out".
}
\par

Revisamos que tipo de archivo es:

\textsf{
>> file testabeatle.out\\
>> testabeatle.out: ELF 64-bit LSB pie executable, x86-64, version 1 (SYSV), dynamically linked,\\
>> interpreter /lib64/ld-linux-x86-64.so.2, for GNU/Linux 2.6.32, BuildID[sha1]=ca68ea305ff7d393662ef8ce4e5eed0b478c8b4e, not stripped
}

Lo analizamos con un \texttt{strings testabeatle.out} y nos un chorrón de texto donde se destaca:

\begin{lstlisting}[frame=single,language=bash,caption={bash version}]
############Challenge#################
Tell me PAul! The result of  5+5?
Ok!ok! it was easy... Tell me now... The result of: 5+5-5*(5/5)?
Last one! The result of: (2.5*16.8+1.25*10.2+40*0.65+1.5*7.5+1.25*3.2):40
Hey Paul! nice!!! this is  the message
VGhlIHRvdXIgd2FzIGNhbmNlbGVkIGZvciB0aGUgZm9sbG93aW5nIG1vbnRoLi4
uIQ0KDQpJJ2xsIGdvIG91dCBmb3IgZGlubmVyIHdpdGggbXkgZ2lybGZyaWVuZC
BuYW1lZCBZb2NvISA7KQ0KDQpIVEJ7UzByUnlfTXlfRlIxM25EfQ0K
WTF! You are not Paul!! SOS SOS SOS HACKER HERE!! I will call the police someone want to steal my data!!!
##########END OF CHALLENGE##################
\end{lstlisting}

Decodeamos el texto en base64 The base64 strings is just the flag:
\texttt{The tour was canceled for the following month...!\\
I'll go out for dinner with my girlfriend named Yoco! ;)\\
HTB{S0rRy\_My\_FR13nD}
}
\par

\textbf{Flag: }HTB\{S0rRy\_My\_FR13nD\}

\end{document}
