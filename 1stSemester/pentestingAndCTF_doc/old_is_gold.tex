%!TeX jobName=challenges/stego/beatles
%pdflatex --jobname=challenges/misc/old_is_gold
%\nofiles
% Created by Bonita Graham
% Last update: February 2019 By Kestutis Bendinskas
%https://es.overleaf.com/gallery/tagged/academic-journal
% Authors:
% Please do not make changes to the preamble until after the solid line of %s.

\documentclass[letterpaper,10pt]{article}
\usepackage[spanish,es-noshorthands]{babel}
\usepackage[latin1,utf8]{inputenc} % Codificación UTF-8

\usepackage[explicit]{titlesec}
\setlength{\parindent}{0pt}
\setlength{\parskip}{1em}
\usepackage{hyphenat}
\usepackage{ragged2e}
%Para colocar puntos en los itemes especiales
\usepackage{pifont}
\RaggedRight

% These commands change the font. If you do not have Garamond on your computer, you will need to install it.
%\usepackage{garamondx}
\usepackage[T1]{fontenc}
\usepackage{amsmath, amsthm}
\usepackage{graphicx}
\usepackage{caption}

% This adjusts the underline to be in keeping with word processors.
\usepackage{soul}
\setul{.6pt}{.4pt}
%This is for the color for the codign
\usepackage[dvipsnames]{xcolor}
%http://latexcolor.com/
\definecolor{almond}{rgb}{0.94, 0.87, 0.8}
\definecolor{airforceblue}{rgb}{0.36, 0.54, 0.66}
\definecolor{antiquewhite}{rgb}{0.98, 0.92, 0.84}
\usepackage{listings}

\lstset{
  backgroundcolor=\color{almond}, %color de fondo
  breaklines=true,
  captionpos=b,     % Establece la posición de la leyenda del cuadro de código
  basicstyle=\sffamily,
  %basicstyle=\footnotesize,
  showstringspaces=false,
  commentstyle=\color{red},
  keywordstyle=\color{blue}
}

%https://www.overleaf.com/learn/latex/Hyperlinks
%Styles and colours
\usepackage{hyperref}
\hypersetup{
    colorlinks=true,
    linkcolor=blue,
    filecolor=magenta,
    urlcolor=ForestGreen,
}
\urlstyle{same}


%%%%%%%%%%%%%%%%%%%%%%%%%%%%%%%%%%%%%%%%%%%%%%%%%%%%
\newcommand{\hlc}[2][antiquewhite]{ \colorbox{#1}{\ttfamily #2} }
%%%%%%%%%%%%%%%%%%%%%%%%%%%%%%%%%%%%%%%%%%%%%%%%%%%%


% The following sets margins to 1 in. on top and bottom and .75 in on left and right, and remove page numbers.
\usepackage{geometry}
\geometry{vmargin={1in,1in}, hmargin={.75in, .75in}}
\usepackage{fancyhdr}
\pagestyle{fancy}
\pagenumbering{gobble}
\renewcommand{\headrulewidth}{0.0pt}
\renewcommand{\footrulewidth}{0.0pt}

% These Commands create the label style for tables, figures and equations.
\usepackage[labelfont={footnotesize,bf} , textfont=footnotesize]{caption}
\captionsetup{labelformat=simple, labelsep=period}
\newcommand\num{\addtocounter{equation}{1}\tag{\theequation}}
\renewcommand{\theequation}{\arabic{equation}}
\makeatletter
\renewcommand\tagform@[1]{\maketag@@@ {\ignorespaces {\footnotesize{\textbf{Equation}}} #1.\unskip \@@italiccorr }}
\makeatother
\setlength{\intextsep}{10pt}
\setlength{\abovecaptionskip}{2pt}
\setlength{\belowcaptionskip}{-10pt}

\renewcommand{\textfraction}{0.10}
\renewcommand{\topfraction}{0.85}
\renewcommand{\bottomfraction}{0.85}
\renewcommand{\floatpagefraction}{0.90}

% These commands set the paragraph and line spacing
\titleformat{\section}
  {\normalfont}{\thesection}{1em}{\MakeUppercase{\textbf{#1}}}
\titlespacing\section{0pt}{0pt}{-10pt}
\titleformat{\subsection}
  {\normalfont}{\thesubsection}{1em}{\textit{#1}}
\titlespacing\subsection{0pt}{0pt}{-8pt}
\renewcommand{\baselinestretch}{1.15}

% This designs the title display style for the maketitle command
\makeatletter
\newcommand\sixteen{\@setfontsize\sixteen{16pt}{6}}
\renewcommand{\maketitle}{\bgroup\setlength{\parindent}{0pt}
\begin{flushleft}
\vspace{-.375in}
\sixteen\bfseries \@title
\medskip
\end{flushleft}
\textsc{\@author}\\
\textit{\today}
\egroup}
\makeatother

% This styles the bibliography and citations.
%\usepackage[biblabel]{cite}
\usepackage[sort&compress]{natbib}
\setlength\bibindent{2em}
\makeatletter
\renewcommand\@biblabel[1]{\textbf{#1.}\hfill}
\makeatother
\renewcommand{\citenumfont}[1]{\textbf{#1}}
\bibpunct{}{}{,~}{s}{,}{,}
\setlength{\bibsep}{0pt plus 0.3ex}

\renewcommand*{\lstlistingname}{Código}  %CAMBIA EL TITULO DE LISTING EN EL CODIGO


%%%%%%%%%%%%%%%%%%%%%%%%%%%%%%%%%%%%%%%%%%%%%%%%%

% Authors: Add additional packages and new commands here.
% Limit your use of new commands and special formatting.

% Place your title below. Use Title Capitalization.
\title{Write-up: HackTheBox - Misc - 0ld is g0ld}

% Add author information below. Communicating author is indicated by an asterisk, the affiliation is shown by superscripted lower case letter if several affiliations need to be noted.
\author{Sebastián Sepúlveda @piblack}


\pagestyle{empty}
\begin{document}

% Makes the title and author information appear.
\vspace*{.01 in}
\maketitle
\vspace{.12 in}

% Start the main part of the manuscript here.
% Comment out section headings if inappropriate to your discipline.
% If you add additional section or subsection headings, use an asterisk * to avoid numbering.

\textbf{Información:} Categoria Misc, 10 points.

\textbf{Descripción:} Old algorithms are not a waste, but are really precious... 

\section*{WriteUp}

Para comenzar con el desafio descargamos el fichero con el comando ya conocido\\

\hlc{wget https://www.hackthebox.eu/storage/challenges/misc/0ld\_is\_g0ld.zip}

Vemos que tiene un pdf! pero que nos depara el destino... Tenemos la opcion que mas nos llama a usar fuerza bruta con rockyou.txt.

\textbf{Como realizamos fuerza bruta a un archivo pdf?}:\\

Para resolver el problema, buscamos en google y nos aparece un programa que se llama \href{http://pdfcrack.sourceforge.net/}{PDFCRACK}. 
En Kali lo descargamos con \hlc{apt-get install pdfcrack}

Pdfcrack lo que hace es fuerza bruta sobre el documento pdf, con algún diccionario que le demos como parámetro. En Kali hay que ingresar el siguiente comando:\\

\hlc{pdfcrack -f ``documento pdf que vamos a atacar'' -w ``diccionario que usaremos''} 

%El número -n va a ser el min de donde comienza a analizar el archivo rockyou.txt y -m el maximo de donde revisa, si no colocamos algun parámetro va a recorrer todos los strings, hasta que encuentre alguno que calce.
En este caso debemos ocupar:

\hlc{pdfcrack -f 0ld\_is\_g0ld.pdf command -w /usr/share/wordlist/rockyou.txt} 

%la fuerza bruta es lo ultimo que deberiamos hacer pero en este caso como es desafio htb, sabemos que rockyou apañaria lo mas probable jeje (o no)
%bueno demosle con el codigo por la shet
%obviuslly we download the app for our desktop
%para hacer un ataque por diccionario necesitamos -w 
%entonces este comando nos sirve para realizar nuestro ataque 
%wuajajaja
%lo instalamos con apt-get install pdfcrack 
%very simple, not?

La clave que encontramos es \texttt{jumanji69}

Abrimos el documento y nos encontramos con una imagen y nada de texto. Sin embargo, al presionar ctrl+A, Eureka!, encontramos algo que no sabemos que es, lo pegamos en un archivo de texto y se muestra puntos y guiones\\

\hlc{.-. .---- .--. ... .- -- ..- ...-- .-.. -- ----- .-. ... ...--}\\

para decodificar el codigo morse ocupamos \href{https://morsecode.scphillips.com/translator.html}{Morse code}

finalmente encontramos la flag.

\textbf{Flag: }HTB\{ripsamu3lmors3\}

\end{document}
